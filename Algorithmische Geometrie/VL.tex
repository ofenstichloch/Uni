\documentclass[a4paper]{article}
\usepackage{amsmath}
\usepackage{amssymb}
\usepackage{fancyhdr}
\usepackage{a4wide}
\usepackage{geometry}
\usepackage[utf8]{inputenc}
\usepackage{graphicx}
\geometry{a4paper,left=2cm,right=2cm, top=3cm, bottom=3cm}
\usepackage{listings}
\lstset{language=xml}
\newcommand{\oh}[1]{$\mathcal{O}(#1)$}
\newcommand{\titel}[1]{\fancyhead[C]{#1}}
\newcommand{\name}{\fancyhead[L]{Alexander Landmesser}}
\newcommand{\matrikel}{\fancyhead[R]{}}
\newcommand{\pl}{\hspace*{1cm}}
\lstset{literate=
  {á}{{\'a}}1 {é}{{\'e}}1 {í}{{\'i}}1 {ó}{{\'o}}1 {ú}{{\'u}}1
  {Á}{{\'A}}1 {É}{{\'E}}1 {Í}{{\'I}}1 {Ó}{{\'O}}1 {Ú}{{\'U}}1
  {à}{{\`a}}1 {è}{{\`e}}1 {ì}{{\`i}}1 {ò}{{\`o}}1 {ù}{{\`u}}1
  {À}{{\`A}}1 {È}{{\'E}}1 {Ì}{{\`I}}1 {Ò}{{\`O}}1 {Ù}{{\`U}}1
  {ä}{{\"a}}1 {ë}{{\"e}}1 {ï}{{\"i}}1 {ö}{{\"o}}1 {ü}{{\"u}}1
  {Ä}{{\"A}}1 {Ë}{{\"E}}1 {Ï}{{\"I}}1 {Ö}{{\"O}}1 {Ü}{{\"U}}1
  {â}{{\^a}}1 {ê}{{\^e}}1 {î}{{\^i}}1 {ô}{{\^o}}1 {û}{{\^u}}1
  {Â}{{\^A}}1 {Ê}{{\^E}}1 {Î}{{\^I}}1 {Ô}{{\^O}}1 {Û}{{\^U}}1
  {œ}{{\oe}}1 {Œ}{{\OE}}1 {æ}{{\ae}}1 {Æ}{{\AE}}1 {ß}{{\ss}}1
  {ű}{{\H{u}}}1 {Ű}{{\H{U}}}1 {ő}{{\H{o}}}1 {Ő}{{\H{O}}}1
  {ç}{{\c c}}1 {Ç}{{\c C}}1 {ø}{{\o}}1 {å}{{\r a}}1 {Å}{{\r A}}1
  {€}{{\EUR}}1 {£}{{\pounds}}1
}
\lstset{language=c++}
\begin{document}
\title{Algorithmische Geometrie}
\maketitle
\section{Konvexe Hüllen}
\subsection{Konvexe Hülle von Punktmengen}
\underline{Definition} Sei $S\subseteq \mathbb{R}^2$ eine Punktmenge in der Ebene. S heißt konvexe Hülle, genau dann, wenn $\forall p,q\in S: \overline{pq} \subseteq S$ wobei $\overline{pq}$ eine Gerade von p nach q ist.\\
Die Konvexe Hülle CH(s) einer Menge $S\subseteq \mathbb{R}^2$ ist die kleinste (in Benzug auf Inklusion) konvexe Menge die S enthält.\\

Eingabe: n Punkte, $S=\{q_1,..,q_n\}$\\
Ausgabe: Ecken $p_1,..,p_k$ der Konvexen Hülle CH(s). Wir wissen, dass $p_i \in S$ für $i=1,..,k$. Ausgabe als Folge gegen den Uhrzeigersinn entlang des Randes von $CH(S)\Rightarrow \overline{p_ip_{i+1}}$ Randsegmente.\\
\underline{Komplexität des Problems}: Satz: Die Berechnung der Konvexen Hülle von n Punkten im $\mathbb{R}^2$ ist mindestens so schwer wie das Sortieren von n reelen Zahlen.\\
Beweis: Reduktion des Sortierens auf CH. Sei CONVEX\_HULL(S) ein ALgorithmus für CH. Zeige, wie man diesen Algorithmus verwenden kann, um n reele Zahlen $x_1,..,x_n$ aufsteigend zu sortieren.\\
Betrachte die Punktmenge $S=\{(x_i,x_i^2)|i=1,...,n\}$. CH(S) liefer alle Punkte in S gegen den Uhrzeigersinn zyklisch sortiert. Wandle die zyklisch sortierte Folge in Linearzeit in eine von "rechts" sortierte Folge um (X-Koordinate aufsteigend sortiert).\\
Folgerung: Die Komplexität des konvexe Hülle Problems ist $\Omega$(n log n) (untere Schranke).
\subsubsection{Gift-Wrapping}
In $\mathbb{R}^3$ oder $\mathbb{R}^2$\\
Idee: Extrempunkt suchen, Strahl anlegen und drehen bis er einen weiteren extremen Punkt erreicht.\\
Lexikographische Ordnung von Punkten p und q $p=(p_x,p_y), p<_{xy} q \Leftrightarrow p_x < q_x \vee (p_x=q_x \wedge p_y<q_y)$.\\
Beobachtung: Der min/max  Punkt in der (xy) oder (yx)-Ordnung ist eine Ecke der konvexen Hülle.\\
Idee für Algorithmus: $S=\{q_1,..,q_n\}$ mit Ecken $p_1,..,p_n$.\\
Startpunkt $p_1\leftarrow min_{xy}(S)$ (unten links).\\
Wie finet man $p_2$: 1. Betrachte horizontalen Strahl nach rechts, der in $p_1$ startet. 2. Drehe diesen gegen den Uhrzeigersinn bis er auf einen Punkt von S trifft. Wiederhole vom so gefundenen $p_2$ aus.\\
Schritt 2 benötigt \oh{n}.\\
Worst-case: h=n (Anzahl Ecken): \oh{n^2}, best-case: h konstant.\\
Details der Implementierung:\\
1. Vergleiche in der linearen Suche: Winkelvergleich, bei Gleichheit Entfernung. Besser: Statt Winkel verwenden wir Orientierung.\\
Definition: Orientation-Prädikat:\\
Gegeben sind drei Punkte $a,b,c \in \mathbb{R}^2$. \\
orientation(a,b,c)=-1 $\Rightarrow$ c liegt rechts der Gerade a,b\\
orientation(a,b,c)=0 $\Rightarrow$ a,b,c liegen auf einer Gerade.\\
orientation(a,b,c)=1 $\Rightarrow$ c liegt links der Gerade a,b\\
%3. Vorlesung (mittwoch 02.11) hier einfügen
\subsubsection{Graham-Scan}
Trick: Lege am Anfang $p_n$ und $p_1$ auf den Stack. Dann haben wir immer $\geq 2$ Punkte auf dem Stack.\\
Der komplette Algorithmus:\\
Vorbedingung: 1. $p_2,...,p_n$ sind aufsteigend nach xy-Ordnung sortiert. 2. $p_i$ liegt oberhalb bzw. auf der Geraden durch $p_2$ und $p_n$
\begin{lstlisting}[escapechar=@]
UPPER_HULL(@$p_1,...,p_n$@)
	Stack S
	s.push(@$p_n$@)
	s.push(@$p_1$@)
	s.push(@$p_2$@)
	for i=3 to n-1 do
		a <- S.top()
		b <- S.top_pred() 
		while (orientation(b,a,@$p_i$@) @$\geq 0$@ do
			S.pop()
			a <- b
			b <- S.top_pred()
		od
		S.push(@$p_i$@)
	od
	return S
\end{lstlisting}
Laufzeitanalyse:\\
Beobachtung: Jeder Punkt wird genau einmal auf den Stack gepusht. Jeder Punkt wird höchstens einmal vom Stack entfernt.\\
$\Rightarrow$ DIe innere Schleife wird insgesamt höchstens n mal ausgeführt. \\
Laufzeit: \oh{n}\\
Analog dazu LOWER\_HULL mit orientation $\leq 0$.\\
Vorbereitung: \\
1. Sortiere die Gesamtmenge S \\
2. Filtere S in S' und S'' (Upper/Lower) \\
3. Berechne UPPER\_HULL(S'), LOWER\_HULL(S'')\\
4. Konstruiere CH(S) aus den beiden Resultaten.\\
Gesamtlaufzeit inklusive Sortieren: \oh{nlog(n)} plus der Rest \oh{n}\\
Satz (Graham):\\
1. Die konvexe Hülle von n Punkten in $\mathbb{R}^2$ kann in Zeit \oh{nlog(n)} berechnet werden.\\
2. Die Laufzeit reduziert sich auf \oh{n}, falls die Punkte sortiert sind.\\
Bemerkungen:\\
1. Varianten: Gleichzeitig obere und untere Hälfte berechnen (2 Stacks). Oder gesamte Hülle in einem zirkulären Scan.\\
2. UPPER\_HULL und LOWER\_HULL sind auch für sich alleine interessant.\\
3. Graham's Scan ist optimal, da die untere Schranke $nlog(n)$ erreicht wird.\\
\subsubsection{Inkrementeller Algorithmus}
Idee: \\
1. Sortiere nach xy-Ordnung\\
2. Betrachte die Punkte nacheinander. Finde obere und untere Tangente von neuem Punkt zu bisheriger Konvexer Hülle, ersetze die damit übersprungenen Kanten mit der jeweiligen Tangente.\\
Finde Berührungspunkte o und u der oberen und unteren Tangente von $p_i$ an das Polygon CH($p_1,...,p_{i-1}$). Entferne alle Ecken zwischen o und u. Füge $p_i$ nach u (vor o) ein. \\
Details: Darstellung der CH: Zirkuläre doppelt verkettete Liste gegen den Uhrzeigersinn, pred-Verweise im Uhrzeigersinn.\\
Initialisierung: CH <- Dreieck ($p_1,p_2,p_3$) unter Beobachtung der Orientierung entweder ($p_1,p_2,p_3$) oder ($p_1,p_3,p_2$). Ein Orientation-Test notwendig.\\
Tangenten und Berührpunkte für $p_i$  mit $i>3$:\\
Verwende Orientation um zu prüfen, ob es Knoten "oberhalb" der Geraden $p_i,p_{i-1}$\\
Pseudocode für oberen Teil:\\
\begin{lstlisting}[escapechar=@]
p <- @$p_{i-1}$@
while !let_turn(@$p_i,p,$@CH.succ(p)) do
	p <- CH.succ(p)
od
\end{lstlisting}
Laufzeit: Beobachtung: Berechnung der Tangenten für einen Punkt $p_i$ hat Laufzeit \oh{1+\#entfernter Ecken}. Gesamtaufwand der Tangentenberechnung linear. \oh{n}
\end{document}

