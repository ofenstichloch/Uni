\documentclass[a4paper]{article}
\usepackage{amsmath}
\usepackage{amssymb}
\usepackage{fancyhdr}
\usepackage{a4wide}
\usepackage{geometry}
\usepackage[utf8]{inputenc}
\usepackage{graphicx}
\geometry{a4paper,left=2cm,right=2cm, top=3cm, bottom=3cm}
\usepackage{listings}

\newcommand{\oh}[1]{$\mathcal{O}(#1)$}
\newcommand{\titel}[1]{\fancyhead[C]{#1}}
\newcommand{\name}{\fancyhead[L]{Alexander Landmesser}}
\newcommand{\matrikel}{\fancyhead[R]{}}
\newcommand{\pl}{\hspace*{1cm}}
\begin{document}
\title{Ausgewählte Kapitel ADS}
\maketitle
\section{Datenstrukturen für Mengen}
\subsection{Union-Find-Problem}
Verwaltung von diskunkten Mengen\\
\subsubsection*{Problem}
Verwalte eine Partition (Zerlegung in disjunkte Teilmengen) der Menge \{1,...,n\} unter folgenden Operationen.\\
Jede Teilmenge (Block) besitzt einen eindeutigen Namen aus \{1,..,n\}.
\begin{itemize}
\item FIND(x): $x\in \{1,..,n\}$ Liefert den Namen der Teilmenge, die x enthält
\item UNION(A,B,C): Vereinigt die Teilmengen mit Namen A und B zu einer Teilmenge mit dem Namen C.
\end{itemize}
\subsubsection*{Initialisierung}
Wir starten mit der Partitionierung: $\{\{1\},..,\{n\}\}$ mit dem Namen $i$ für $\{i\}$,$1\leq i \leq n$\\
Analyse: Kosten für 1 Union (worst case)\\
Amortisiert: Kosten für $n-1$ mögliche UNIONs\\
$\rightarrow$ Kosten von $n-1$ UNIONs und m FINDs\\
\underline{\textbf{1. Lösung} (einfach)}\\
Verwende ein Feld name[1..n] mit name[x] = Name des Blocks der x enthält. $1\leq x \leq n$\\
\begin{lstlisting}
for i=1 to n do
	name[i] <- i
done
\end{lstlisting}
FIND(x): return name[x] : \oh{n}\\
UNION(A,B,C): \oh{n}
\begin{lstlisting}
for i=1 to n do
	if name[i] = A OR name[i] = B
	then name[i] <- C
	fi
done
\end{lstlisting}
Gesamtlaufzeit (Lemma 1):\\
$n-1$ UNIONs und m FINDs kosten \oh{n^2+m}\\
\underline{\textbf{2. Lösung} (Verbesserung)}\\
1. Find unverändert\\
2. Ändere den Namen der kleineren Menge in den Namen der größeren\\
Zusätzliche Felder:
\begin{itemize}
\item size[1..n]: size[A] = Anzahl Elemente im Block A, initialisiert mit 1
\item L[1..n]: L[A] = Liste aller Elemente in Block A, initialisiert L[i] = \{i\}
\end{itemize}
FIND(x) bleibt gleich\\
UNION(A,B):
\begin{lstlisting}[escapechar=!]
if size[A] !$\leq$! size[B]
then
	forall i in L[A] do
		name[i]!$\leftarrow$! B
	done
	size[B] += size[A]
	L[B] !$\leftarrow$! L[B] concatenate L[C]
else
	symmetrisch
\end{lstlisting}
Die Menge heißt jetzt A oder B\\
Effekt: UNION(A,B,..) hat Laufzeit \oh{min(|A|,|B|)}\\
Worst Case eines UNION dieser Folge von UNIONs: \oh{\frac{n}{2}} = \oh{n} (kann nur einmal vorkommen)\\
Gesamtkosten für alle n-1 UNIONs: \oh{n*log(n)}
\end{document}