\documentclass[a4paper]{article}
\usepackage{amsmath}
\usepackage{amssymb}
\usepackage{fancyhdr}
\usepackage{a4wide}
\usepackage{geometry}
\usepackage[utf8]{inputenc}
\usepackage{graphicx}
\geometry{a4paper,left=2cm,right=2cm, top=3cm, bottom=3cm}
\usepackage{listings}
\lstset{language=xml}
\newcommand{\oh}[1]{$\mathcal{O}(#1)$}
\newcommand{\titel}[1]{\fancyhead[C]{#1}}
\newcommand{\name}{\fancyhead[L]{Alexander Landmesser}}
\newcommand{\matrikel}{\fancyhead[R]{}}
\newcommand{\pl}{\hspace*{1cm}}
\lstset{literate=
  {á}{{\'a}}1 {é}{{\'e}}1 {í}{{\'i}}1 {ó}{{\'o}}1 {ú}{{\'u}}1
  {Á}{{\'A}}1 {É}{{\'E}}1 {Í}{{\'I}}1 {Ó}{{\'O}}1 {Ú}{{\'U}}1
  {à}{{\`a}}1 {è}{{\`e}}1 {ì}{{\`i}}1 {ò}{{\`o}}1 {ù}{{\`u}}1
  {À}{{\`A}}1 {È}{{\'E}}1 {Ì}{{\`I}}1 {Ò}{{\`O}}1 {Ù}{{\`U}}1
  {ä}{{\"a}}1 {ë}{{\"e}}1 {ï}{{\"i}}1 {ö}{{\"o}}1 {ü}{{\"u}}1
  {Ä}{{\"A}}1 {Ë}{{\"E}}1 {Ï}{{\"I}}1 {Ö}{{\"O}}1 {Ü}{{\"U}}1
  {â}{{\^a}}1 {ê}{{\^e}}1 {î}{{\^i}}1 {ô}{{\^o}}1 {û}{{\^u}}1
  {Â}{{\^A}}1 {Ê}{{\^E}}1 {Î}{{\^I}}1 {Ô}{{\^O}}1 {Û}{{\^U}}1
  {œ}{{\oe}}1 {Œ}{{\OE}}1 {æ}{{\ae}}1 {Æ}{{\AE}}1 {ß}{{\ss}}1
  {ű}{{\H{u}}}1 {Ű}{{\H{U}}}1 {ő}{{\H{o}}}1 {Ő}{{\H{O}}}1
  {ç}{{\c c}}1 {Ç}{{\c C}}1 {ø}{{\o}}1 {å}{{\r a}}1 {Å}{{\r A}}1
  {€}{{\EUR}}1 {£}{{\pounds}}1
}
\lstset{language=c++}
\begin{document}
\title{Netzwerkalgorithmen}
\maketitle
\section{Zusätzliches blabla}
Makros in C/C++: \#define alias replace, wobei replace auch Code sein kann.\\

\section{Datentypen für Graphen und Netzwerke (LEDA)}
\subsection*{Definition eines Datentyps}
Definition der Objekte des Typs: $stack<T>$ \\
Konstruktion: $stack<int>$ S(100) (max Größe)\\
Operationen: $s.push(T x)$, $T s.pop()$\\
Bemerkung zu Implementierung\\
\subsection*{Graph-Datentyp in LEDA}
Der Typ graph repräsentiert gerichtete Graphen.\\
Ein Graph g besteht aus zwei Typen von Objekten: node und edge\\
Mit jedem Knoten v sind zwei Listen von Kanten ($list<edge>$) verbunden (eingehend und ausgehend)\\
Mit jeder Kante e werden 2 Knoten source und target gespeichert.\\
\subsubsection*{Operationen auf G}
Update:\\
\pl node G.new\_node(), erzeugt einen neuen Knoten in G und gibt ihn zurück.
\pl edge G.new\_edge(node v, node w)\\
\pl void G.del\_edge(edge)\\
Access:\\
\pl $list<edge>$ G.out\_edges(node v);\\
\pl int G.outdeg(node v);\\
\pl node G.source(edge);\\
\pl node G.target(edge);\\
Iteration:\\
\pl forall\_nodes(v,G)\\
\pl forall\_edges(e,G)\\
\pl forall\_out\_edges(e,v)\\
\pl forall\_in\_edges(e,v)\\
\subsubsection*{1. Problem}
Gegeben: Graph G=(V,E)\\
Frage: Ist G azyklisch?\\
Algorithmus siehe Topologisches Sortieren: Entferne jeweils einen Knoten v mit indeg(v)=0 bis der Graph leer ist. Falls wir keinen solchen Knoten finden dann ist der Graph zyklisch, falls G am Ende leer, ist er azyklisch.\\
C++:\\
\begin{lstlisting}
bool ACYCLIC(graph G){ 		//Call by value damit G nicht zerstört
	list<node> zero;
	node v;
	forall_nodes(v,G){
		if (G.indeg(v)==0) zero.append(v);
	}
	while (!zero.empty()){
		node u = zero.pop();
		edge e;
		forall_out_edges(e,u){
			node w=G.target(e);
			G.del_edge(e);
			if (G.indeg(w) == 0){
				zero.append(w);
			}
		}
	}
	return G.empty();
}
\end{lstlisting}
\subsubsection*{Daten für Knoten und Kanten}
1. Parametrisierte Graphen: GRAPH$<$node\_type,edge\_type$>$ G\\
2. Temporäre Daten: besucht[v] $\leftarrow$ true
\subsection*{Datentypen in LEDA}
node\_array$<T>$ A(G,x): Feld über die Knoten des Graphen G
edge\_array$<T>$ B(G,y) analog
Verwendet für: Temporäre Daten, Eingabedaten, Resultate\\
\subsubsection*{Anwendung im topologischen Sortieren}
injektive Abbildung: topnum: $V\rightarrow \{1,...,n\}$ mit $\forall (v,w)\in E$: topnum[v]$<$topnum[w]\\
\begin{lstlisting}
bool TOPSORT(const graph& G, node_array<int>& topnum){
	int count = 0;
	list<node> zerol
	node_array<int> indeg(G);
	node v;
	forall_nodes(v,G){
		indeg[v] = G.indeg(v);
		if(indev[v] == 0) zero.append(v);
	}
	while(!zero.empty()){
		node v = zero.pop();
		topnum[v] = ++count;
		edge e;
		forall_out_edges(e,v){
			node w = G.target(e)
			if (--indeg[w] == 0) zero.append(w)
		}
	}
	return count == G.number_of_nodes();
}
\end{lstlisting}
\subsubsection*{Tiefensuche}
Hauptprogramm:
\begin{lstlisting}
void DFS(const graph& G, node_array<int>& dfsnum, node_array<int>& compnum){
	int count1 = 0;
	int count2 = 0;
	node_array<bool> visited(G,false);
	node v;
	forall_nodes(v,G){
		if (!visited[v]) dfs(G,v,count1,count2,dfsnum,compnum)
	}
}
\end{lstlisting}
Rekursive Funktion dfs:
\begin{lstlisting}
void dfs(const graph& g, node v, int& count1, int& count2,node_array<int>& dfsnum, _
		node_array<int>& compnum){
	dfsnum[v] == ++count1;
	visited[v] = true;
	edge e;
	forall_out_edges(e,v){
		edge w = G.target(e),
		if(!visited[w]) dfs(G,w,count1,count2,dfsnum,compnum)
	}
	compnum[v] = ++count2
}
\end{lstlisting}
\subsubsection*{Berechnung starker Zusammenhangskomponenten}
Definition: Ein gerichteter Graph ist stark zusammenhängend, wenn $\forall v,w \ in V: v\rightarrow^* w$ (es existiert ein Pfad von v nach w)\\
Die starken Zusammenhangskomponenten (SZK) von G sind die maximalen SZK Teilgraphen von G.\\
Idee für Algorithmus:\\
\begin{enumerate}
\item führe DFS mit $G'=(V',E')$ dem Teilgraphen aufgespannt von bereits besuchten Knoten
\item Verwalte SZK von $G'$ während DFS ausgeführt wird.
\end{enumerate}
Ablauf:\\
Sei (v,w) die nächste in dfs betrachtete Kante\\
1. Fall: $(v,w)\in T$ (Baumkante), w noch nicht besucht.  $V' = V'\cup \{w\}, E' = E'\cup \{(v,w)\},SZK=SZK\cup \{\{w\}\}$\\
2. Fall: $(v,w)\not\in T$,d.h. w wurde schon besucht, ist also in $V'$ enthalten.$E'=E'\cup\{(v,w)\}$. Nun kann (v,w) meherere bereits bekannte SZK vereinigen (Rückwärtskante oder Cross-Kante). Bemerkung: Vorwärtskanten generieren keine neuen Pfade in $G'$, deswegen dabei keine änderung der SZK.\\
Bezeichnungen:\\
\begin{enumerate}
\item Eine SKZ K heißt abgeschlossen, falls die Aufrufe von dfs für alle Knoten v in K beendet sind.
\item Die Wurzel einer SZK K ist der Knoten mit der kleinsten dfsnum in K.
\item "Unfertig" ist die Folge aller Knoten für die dfs aufgerufen wurde, aber deren SZK in der sie sich befinden noch nicht abgeschlossen ist.
\item "Wurzeln" ist die Folge aller Wurzeln der nicht abgeschlossenen SZK nach dfsnum sortiert.
\end{enumerate}
Situation, wenn DFS beim Knoten g angekommen ist:\\
%TODO Beispiel hier
Unfertig: a,b,c,e,f,g\\
Wurzeln: 1,b,e,g\\
Der Algorithmus betrachtet danach dei Kanten aus g $(g,d)\in C$: es passiert nichts, da d in einer abgeschlossenen SZK ist;$(g,c)\in C$ Vereinigt die 3 SZK mit den Wurzeln b,e,g durch entfernen von e,g aus der Wurzelfolge.\\
Beobachtung: Hinzufügen und Streichen nur am Ende $\rightarrow$ Stack eignet sich als Datenstruktur.\\
Allgemeine Situation für $(v,w)\in T$:\\
%TODO Bild
$K' = K_2\cup K_3\cup K_4$.\\
\begin{tabular}{c|c|c|c}
$K_1$&$K_2$&$K_3$&$K_4$\\
$r_1$|||&$r_2$|||&$r_3$|||&$r_4$|||\\
\end{tabular}\\
Ergänzungen von DFS für SZK:\\
1. Aktion: while dfsum[wurzel.top()] $>$ dfsnum[w] do wurzeln.pop() od\\
2. Falls $(v,w)\in T$: Wurzeln.push(w); Unfertig.push(v)\\
3. Abschluss eine SZK: SZK von v wird endgültig verlassen, sie ist nun abgeschlossen.\\
\pl Am Ende von dfs(v): if v == Wurzel.top() then \\
\pl \pl Wurzeln.pop();\\
\pl \pl repeat w = unfertig.pop() until w==v\\
\pl fi\\
Übung 2: Algo zu SZK.\\
Darstellung der einzelnen SZKs: node\_array$<int>$ szknum(G); k=\#kompnum $forall v\in V szknum[v] = i \Leftrightarrow $v in der Komponente i\\
$\rightarrow$  int SZK(const graph\& G, node\_array$<int>$\& sznum)\\
\section{Kürzeste Wege / Billigste Wege}
Allgemeines Problem: Sei gegeben ein gerichteter Graph $G=(V,E)$, eine Kostenfunktion $cost:E\rightarrow \mathbb{R}$ mit $cost(e)=$Kosten der Kante e.\\
Beispiel:\\
%TODO Bild
\includegraphics[scale=0.05]{3.jpg} k=3\\
Kosten eines Pfades P: $cost(P):= \sum_{e\in P} cost(e)$\\
\textbf{Billigste-Wege Problem:}Finde einen Pfad P mit cost(P) minimal
\begin{itemize}
\item[a)] Zwischen zwei gegebenen Knoten
\item[b)] Von einem Knoten s zu allen anderen Knoten (single source shortest paths)
\item[c)] Zwischen allen Paaren $(v,w)\in V\times V$
\end{itemize}
Im Beispiel: $dist(a,f) = 2$, $dist(a,e) = -\infty$ weil $a\rightarrow b \rightarrow (c\rightarrow d\rightarrow b\rightarrow c)^i \rightarrow e$\\
Definition: $dist(v,w) := inf\lbrace cost(P)|P \text{ist Pfad von v nach w}\rbrace$ d.h. wenn es einen negativen Zyklus gibt = $-\infty$, wenn kein Pfad existiert: $\infty$.\\
Wir betrachten hier die Single-Source Variante:\\
Eingabe: $G=(V,E), s\in V, cost:E\Rightarrow\mathbb{R}$\\
Ausgabe: $\forall v\in V: dist(s,v)$\\
Beobachtung: Die dist-Funktion erfüllt die Dreiecksungleichung: $dist(s,w)\leq dist(s,v)+cost(v,w)$\\
Idee für allgemeinen Algorithmus (Label Correcting):\\
Überschätze die dist-Werte (DIST=$\infty$). Solange eine Kante (u,v) die Dreiecksungleichung verletzt, korrigiere DIST[v]\\
\begin{lstlisting}[escapechar=!]
forall v in V do
	DIST[v] = !$\infty$!
od
DIST[s] = 0
while !$\exists (u,v)\in E$! mit !$DIST[v]>DIST[u]+cost(u,v)$!
	DIST[v] = DIST[u]+cost(u,v)
od
\end{lstlisting}
Eigenschaften
\begin{itemize}
\item[a)] Es gilt immer $DIST[v] \geq dist(s,v)$
\item[b)] Wenn $DIST[v] < \infty$, dann existiert ein Pfad von s nach v mit den Kosten DIST[v]
\item[c)] Die kürzesten Pfade bilden einen Baum T mit Wurzel s. Begründung: In jeden Knoten mit $DIST[v] < \infty$ mündet genau ein kürzester Pfad. Bei korrigieren von DIST-Wert wird die eingehende Kante von T definiert.
\item[d)] Für jede Kante $(v,w)$ auf einem Kürzesten Pfad gilt: $dist(s,w) = dist(s,v)+cost(v,w)$
\end{itemize}
Da DIST[v] $<\infty$ existiert ein Pfad von s nach v mit cost(P)=DIST[v]. Daraus folgt DIST[v]$\geq$ dist(s,v), da dist(s,v)=inf\{cost(P) mit P ist Pfad von s nach v\}\\
Effizienter Algorithmus: Kandidatenmenge\\
Wir speichern alle Knoten aus denen Kanten ausgehen können, die die Dreiecksungleichung verletzen in einer Menge U.\\
Am Anfang gilt: $U=\{s\}$. Immer wenn DIST[v] vermindert wird, nimm v nach U auf.
Verfeinerter Algorithmus:
\begin{lstlisting}[escapechar=!]
1. forall v !$\in$! V do
	DIST[v] = !$\infty$!
   od
2. DIST[s] = 0
3. U = {s}
4. while !$U\neq \emptyset$!
5. 	wähle und entferne ein u !$\in$! U
6	forall !$v\in V$! mit !$(u,v)\in E$! do
7.	  c = DIST[u]+cost(u,v)
8.	  if c < DIST[v] then
9.	  	DIST[v]=c
10.		U = U!$\cup$! {v}
11.	  fi
12.	od
13.od
\end{lstlisting}
Eigenschaften des Algorithmus während Laufzeit (Lemma):\\
Falls der Graph G keine negativen Zyklen enthält, dann gilt: \\
a) Falls $v\not\in U$, dann gilt für alle ausgehenden Kanten (v,w): DIST[w]$\leq$ DIST[v]+c(v,w).\\
b) Sei $v_0,...,v_k$ ein billigster Pfad von s nach v. Falls nun DIST[v]$>$ dist(s,v), dann existiert ein i $0\leq i \leq k-1$ mit DIST[$v_i$] = dist(s,$v_i$) und $v_i\in U$\\
c) Es existiert immer ein $u\in U$ mit DIST[u]=dist(s,u) und wenn in Zeile 5 des Algorithmus ein solches u gewählt wird (perfekte Wahl), dann wird die while-Schleife für jeden Knoten höchstes einmal ausgeführt.\\
Beweis: \\
a) Induktion über Schleifendurchläufe i (while)\\
i=0: Vor dem ersten Durchlauf gilt die Behauptung, da DIST[s]=0 und DIST[v]=$\infty\forall v\in V$, U=$\{s\}$\\
i=i+1: Betrachte ein beliebiges $v\not\in U$ nach der i+1-ten Ausführung\\
\pl $v\not\in U$ vor der (i+1)-ten Ausführung -> Nach IA: DIST[v]+c(v,w)$\geq$ DIST[w] für alle ausgehenden Kanten (v,w) und DIST[v] wurde in diesem Durchlauf nicht verändert. Die  DIST-Werte der Nachbarn w von v wurden eventuell verändert $\Rightarrow DIST[v]+c(v,w) \geq DIST[w]$ für alle ausgehenden Kanten \\
\pl $v\in U$ vor der (i+1)-ten Iteration: v wurde in Zeile 5 ausgewählt, dann stellt die innere Schleife die Dreiecksungleichung für alle ausgehenden Kanten wieder her.\\
b) Sei $s=v_0,...,v_k=v$ ein billigster Pfad von s nach v mit $DIST[v_k]>dist(s,v_k)$\\
Sei i maximal mit $DIST[v_i] = dist(s,v_i) \Leftarrow 0\leq i<k$ ($v_0$ hat korrekten Wert)\\
Wir zeigen $v_i\in U$:\\
Annahme $v_i \not\in U$: $DIST[v_i]+c(v_i,v_{i+1})\geq DIST[v_{i+1}]$ Außerdem gilt: dist(s,$v_{i+1}$) = $dist(s,v_i)+c(v_i,v_{i+1})$.\\
Daraus folgt: $dist(s,v_{i+1}) = dist(s,v_i)+c(v_i,v_{i+1}) = DIST[v_i]+c(v_i,v_{i+1}) \geq DIST[v_{i+1}]\Rightarrow dist(s,v_{i+1}) = DIST[v_{i+1}]$ Widerspruch zur Wahl von i.\\
c) Sei $v\in U$ beliebig\\
Falls DIST[u] = dist(s,u) ok.\\
Falls DIST[u] $>$ dist(s,u): Betrachte einen billigsten Pfad von s nach u, auf dem nach b) ein $v_i\in U $ existiert, mit DIST[$v_i$] = dist(s,$v_i$)\\
Perfekte Wahl: Falls in Zeile 5 immer ein solcher Knoten $u\in U$ gewählt wird, dann kann u kein zweites Mal nach U angefügt werden.\\
Perkefte Wahl $\Rightarrow$ Gesamtaufwand: \oh{\sum_{v\in V}(1+outdeg(v)+Verwaltung\ der\ Menge\ U} = $n+m+?$\\
1. Allgemeiner Fall: beliebiger Graph mit beliebiger cost-fUnktion: Realisiere U als Schlange (FIFO)\\
Keine Negativen Zyklen $\Rightarrow$ in der Schlange existiert mindestens ein perfekter Knoten $\Rightarrow$ Zwischen 2 aufeinanderfolgenden Entnahmen des selben Knoten u wurde mindestens ein perfekter Knoten entfernt. $\Rightarrow$ Jeder Knoten wird maximal n-mal entfernt (spätestens dann ist U leer).\\
Beobachtung: Wird ein Knoten öfters als n-mal entfernt, existiert ein negativer Zyklus\\
Bellman/Ford Algorithmus:
\begin{lstlisting}
bool BELLMAN(const graph& G, node s, const edge_array<int>& cost, node_array<int>& DIST){
	queue<node> Q; \\Menge U
	node_array<bool> inQ(G,false);
	node_array<int> count;
	DIST[s]=0;
	Q.append(s);
	inQ[s] = true;
	while (!Q.empty()){
		node u = Q.pop()
		inQ[] = false;
		if (++count[u] > G.number_of_nodes()){
			return false;
		}
		edge e;
		forall_out_edges(e,u){
			node v=G.target()
			int c = DIST[u]+cost[e]
			if (c<DIST[v]){
				DIST[v] = c; 
				//PRED würde hier gesetzt
				if (!inQ[v]){
					Q.append(v)
					inQ[v]=true
				}
			}
		}
	}
	return true;
}
\end{lstlisting}
Laufzeitanalyse: n*(Iterationen über alle Knoten + deren ausgehenden Kanten) = \oh{\sum_{v\in V}(1+outdeg(v))}\\
Gesamt: \oh{n^2+n*m}\\
Wenn G zusammenhängend gilt $m\geq n-1 \Rightarrow$ \oh{n*m}\\
Problem: Falls negativer Zyklus existiert, gebe einen oder alle aus.
2. Azyklische Graphen: Es existiert eine topologische Sortierung.\\
Behauptung: Der Knoten $u\in U$ mit kleinster topologischer Nummer ist eine perfekte Wahl mit DIST[u] = dist(s,u)\\
Beweis: Sei $u\in U$ mit topnum[u] minimal\\
\pl Annahme: DIST[u] $>$ dist(s,u)
\pl $\Rightarrow \exists$ Knoten v auf dem kürzesten Pfad von s nach u mit DIST[v] = dist(s,u) und $v\in U$\\
\pl $\Rightarrow $ topnum[v] $<$ topnum[u] Widerpsruch.\\
3. Nicht-Negative Netzwerke: Beliebiger Graph aber Kosten $\geq 0$\\
Dijksra...\\
Wähle in Zeile 5 einen Knoten $u\in U$ mit DIST[u] minimal. $\Rightarrow$ Priority-Queue.\\
Behauptung: $u\in U$ mit DIST[u] minimal ist perfekt d.h. DIST[u] = dist(s,u)\\
Beweis:
\pl Annahme: DIST[u] $>$ dist(s,u) $\Rightarrow \exists v\in U$ mit DIST[v] = dist(s,v) auf einem kürzesten Pfad von s nach u.\\
\pl Dann gilt: dist(s,u) $\geq_{nichtNegativ}$ dist(s,u) $=_{Lemma}$ DIST[v] $\geq_{minmalwahl} $ DIST[u]\\
\pl $\Rightarrow$ dist(s,u) = DIST[u] da DIST-Label nie zu klein werden.\\

Darstellung der billigsten Pfade durch pred-Verweise: Merke für jeden Knoten die eingehende Kante des kürzesten Weges.
\begin{lstlisting}
Def: node_array<edge> PRED
Init: foreall_nodes(v,G) PRED[v] = NULL
in der inneren Schleife (forall_out_Edges(e,u): if(c<DIST[v]){ DIST[v] = c; PRED[v] = e}
Durchlaufen der Pfade:
while(PRED[v] != NULL){
	edge e=PRED[v]
	print e 
	v=G.source(e)
}
\end{lstlisting}

Dijkstra-Algorithmus: Wähle Koten u mit DIST[u] minimal.\\
Datenstruktur: Priority Queue\\
LEDA: $node_pq<P> PQ$ Knoten-Priority-Queue.\\
Operationen: 
\begin{itemize}
\item Konstruktor $node_PQ<P> PQ(Graph G)$ Mit leerer PQ für Knoten von G\\
\item void insert (node v,P p)
\item node PQ.del\_min() entfernt einen Knoten v mit kleinster Priorität und gibt diesen zurück
\item node PQ.find\_min() 
\item void PQ.decrease\_p(node v P p) vermindert die Priorität von v auf p
\item bool PQ.empty()
\end{itemize}
\begin{lstlisting}
void DIJKSTRA(const graph& G, node s, const edge_array<int>& cost, node_array<int>& dist, node_array<edge>& pred){
	node_pq<int> PQ(G);
	node v;
	forall_nodes(v,G){
		dist[v] = MAXINT
		pred[v] = NULL
	}
	dist[s] = 0
	PQ.insert(s,0);
	while(!PQ.empty()){
		node u = PQ.del_min()
		edge e
		forall_out_edges(e,u){
			node v = G.target(e)
			int c = dist[u]+cost[e]
			if(c < dist[v]){
				if(dist[v] == MAXINT){
					PQ.insert(v,c)
				}else{
					PQ.decrease(v,c)
				}
				dist[v] = c
				pred[v] = e
			}
		}
	}
}
\end{lstlisting}
Laufzeitanalyse: Operationen auf dem Graphen \oh{n+m}. PQ-Operationen: 1x Konstruktor, nx insert, nx del\_min nx empty, $leq mx$ decrease\\
Gesamtlaufzeit: \oh{n*(T_{insert}+T_{delmin}+T_{empty})+m*T_{decrease}}\\
Varianten Datentyp: binärer min-heap, balancierter Baum, Fibonacci-Heap\\
Varianten Verfahren: All-Pairs-Problem, Single-source-signle-target\\

\section{MaxFlow}
Transport-Problem: Wie viele Einheiten können von s nach t in einer Zeiteinheit transportiert werden. Jede Kante hat eine Kapazität, die Menge die pro Zeiteinheit über diese Kante laufen kann.\\
Definition (Network flows): Gegeben sei ein gerichteter Graph $G=(V,E)$, eine Kapazitätsfunktion $u:E\rightarrow \mathbb{R}^+_0$ und zwei Knoten $s,t\in V$ mit $s\neq t$. $u((v,w))$ heißt Kapazität von (v,w). s heißt Quelle(Source), t heißt Senke(target).\\
Gesucht: Flussfunktion $x:E\rightarrow \mathbb{R}^+_0$ mit den Eigenschaften:\\
\pl 1. $\forall e\in E: 0\leq x(e)\leq u(e)$ Kapazitätsbedingung\\
\pl 2. Sei $v\in V: \sum_{(v,u)\in E} x(v,u) - \sum_{(w,v)\in E} x(w,v) = \{ F: v=s; 0: v\not\in \{s,t\};-F:v=t\}$ Mit dem Fluss F. (Massenbalancebedingung)\\
\pl 3. F soll maximal sein.\\
Bemerkung mit F=0 ist das Problem trivial.\\
Notation: Links der Kante steht die Kapazität, rechts der Fluss.\\
Knoten \{1,...,n\} Kanten (i,j), Kapazitäten $u_{ij}$, Flüsse $x_{ij}$\\
\textbf{Definition:} Restnetzwerk (residual network): beschreibt mögliche Flussrichtungen. Sei x eine aktuelle Flussfunktion, d.h. x erfüllt die Kapazitätsbedingungen (z.B. Nullfluss x=0). Das Restnetzwerk G(x) besteht aus allen Knoten von G und folgenden Kanten: Für jede Kante (i,j) in G existieren in G(x) 2 Kanten (i,j) und (j,i) mit Restkapazitäten $r_{ij} = u_{ij} - x_{ij}$ und  $r_{ji} = x_{ij}$. Beschreibt mögliche Flussänderungen der Flussfunktion, ingnoriert of alle Kanten mit Restkapazität 0.\\
%Graph beispiel
\includegraphics[scale=0.1]{4.jpg}\\
Beobachtung: Jeder Pfad in G(x) von s nach t beschreibt eine mögliche Erhöhung des Gesamtflusses f.\\
Im Beispiel, ohne 0-Kanten, existieren 3 Pfade, 1-3-2-4 mit den Restkapazitäten 1,2,1. Die mögliche Erhöhung ist das Minimum dieser Pfade.\\
Wenn in G(x) ohne 0-Kanten kein Pfad von s nach t existiert, ist keine Erhöhung möglich $\Rightarrow$ f maximal.\\
Für x = 0 gilt: G(x)=G\\
Saturation einer Kante (i,j) in G $\Rightarrow x_{ij}\leftarrow u_{ij}$ entfernt (i,j) aus G(x)\\
Auf Null setzen $x_{ij}\leftarrow 0$ entfernt die Gegenkante (i,j) aus G(x).\\
Bemerkung: Effiziente Implementierungen verwenden später keinen expliziten Restgraphen G(x).\\
\textbf{Begriff des Schnitts} CUT:\\
intuitiv: Betrachte einen Schnitt der s und t trennt\\
Alles was von s nach t fließt, fließt über diesen Schnitt $\rightarrow$ Kapazität des Schnittes ist obere Schranke, das gilt für alle Schnitte und damit auch den minimalen Schnitt.\\
\textbf{Formale Definition:} (s,t)-Schnitt: Ein (s,t)-Schnitt $[S,\overline{S}]$ ist eine Partitionierung der Knoten V in 2 disjunkte Teilmengen $S$ und $\overline{S}=\{V\setminus S\}$ mit $s\in S$ und $t\in \overline{S}$. NOtation: $(i,j)\in (S,\overline{S})$\\
Fluß von s nach t muss über die Kanten ($S,\overline{S}$). \\
Definition: Kapazität von $[S,\overline{S}]: u[S,\overline{S}] = \sum_{(i,j)\in [S,\overline{S}]} u_{ij}$. Analog Restkapazität $r[S,\overline{S}]=\sum_{(i,j)\in [S,\overline{S}]}$. Fluss über den Schnitt $f=\sum_{(i,j)\in [S,\overline{S}]}x_{ij} -\sum_{(i,j)\in [\overline{S},S]}x_{ij}$, wobei $\sum_{(i,j)\in [S,\overline{S}]}x_{ij}$ maximal $\sum_{(i,j)\in [S,\overline{S}]}u_{ij}=u[S,\overline{S}]$ ist.\\
Für alle (s,t)-Schnitte gilt $f\leq u[S,\overline{S}]$, insbesondere für den mit minimaler Kapazität. $\Rightarrow$ Der Wert eines maximalen Flusses f ist nicht größer, als die Kapazität eines minimalen Schnittes. ($f_{max}\leq minCut$, schwache Variante von MaxFlowMinCut)\\
Restnetzwerkberechnen: Explizit teuer, besser implizit.\\
Algorithmus zur Berechnung des Restnetzwerks:\\
\begin{lstlisting}
forall_out_edges(e){
	r = cap[e]-flux[e]
	if r==0 continue
	baue Graph
}
forall_in_edges(e){
	r=flow[e]
	if r==0 continue
\end{lstlisting}
Algorithmus für MaxFlow (Labeling-Algorithmus):
1. x=0\\
2. while G(x) enhält einen Pfad von s nach t do\\
3.   Sei P ein solcher erhöhender Pfad\\
4.   $S=min\{r_{ij}| (i,j)\in P\}$\\
5.   Erhöhe den Fluss x entlang von P um r Einheiten.\\
6.   berechne G(x) neu.\\
7. od\\
Idee für Implementierung: Sei S die Menge der von s aus erreichbaren Knoten. Wenn t markiert wurde gibt es einen Pfad von s nach t und t liegt in S. Pfade werden in Pred-Array gespeichert.
\begin{lstlisting}
void MF_Labeling(const graph& G, node s, node t, const edge_array<int>& cap, edge_array<int>& flow){
	list<node> L; //Menge S
	node_array<bool> labeled (G,false)
	node_array<edge> pred (G,Null)
	while(true){
		labeled[s]=true
		L.append(s)
		while(!L.empty()){
			node v = L.pop()
			edge e
			forall_out_edges(e,v){
				if(flow[e]==cap[e]) continue
				node w = G.target(e)
				if(labeled[w]) then continue
				labeled w = true
				pred[w]=e
				L.append(w)
			}
			forall_in_edges(e,v){
				if(flow[e]==0) continue
				node w = G.source(e)
				if(labeled[w]) then continue
				labeled w = true
				pred[w]=e
				L.append(w)
			}
			if (labeled[t]) L.clear()
		}
		if (labeled[t]) AUGMENT(G,s,t,pred,cap,flow) //Pfaderhöung
		else break
	}

}
void AUGMENT( const graph& G, node s, node t, const node_array<edge>& pred, const node_array<int>& cap, const node_array<int>& flow){
	int delta = MAXINT
	node v = t
	while (v != s){
		int r
		edge e = pred[v]
		if(v == G.source(e)){
			r = flow[e]
			v = G.target(e)		
		}else{
			r = cap[e]-flow[e]
			v = G.source(e)
		}
		if(r<delta) delta = r
	}
	v=t
	while(v!=s){
		edge e = pred[v]
		if (v==G.source(e)){
			flow[e] -= delta
			v = G.target(e)
		}else{
			flow[e] += delta
			v=G.source(e)
		}
	}
}
\end{lstlisting}

\subsection{Labeling-Algorithmus}
\subsubsection*{Korektheit}
In jedem Schritt (Hauptschleife) gibt es 2 Möglichkeiten:\\
a) Algorithmus findet einen erhöhenden Pfad. $\rightarrow$ AUGMENT\\
b) Kein erhöhender Pfad (t wird nicht gelabeled) $\rightarrow$ STOP\\
Zu zeigen: Im Fall b ist der berechnete Fluss maximal.\\
Knoten die gelabeled sind ($S$) und solche die nicht gelabeled sind ($\overline{S}$) ist ein (s,t)-Schnitt.\\
Beobachtung: $\forall (i,j)\in [S,\overline{S}]$ gilt: $v_{ij}=0$ (deshalb werden diese Kanten vom Algorithmus nicht mehr benutzt, um weitere Knoten zu labeln.\\
Bei uns bedeutet dies: $\forall$ Kanten $e \in (S,\overline{S}):flow[e]==cap[e]$ und $\forall e\in (\overline{S},S): flow[e]=0$\\
Beobachtung über Flüsse und Schnitte: $F=\sum x_{ij} - \sum x_{kl} (ij)\in (S,\overline{S}), (k,l)\in (\overline{S},S)$\\
Hier gilt also: $F=\sum u_{ij} - \sum 0 = u[S,\overleftarrow{S}]$\\
Aus der schwachen Version des Maxflow/MinCut Theorems ($F_{max} \leq U_{min}$) folgt F ist maximal und $u[S,\overline{S}]$ minimal.\\
Allgemein: Der Algorithmus liefert zusätzlich zur optimalen Lösung des primalen Problems (Maxflow) eine optimale Lösung des dualen Problems (MinCut).\\
\textbf{Satz (MaxFlow-MinCut Theorem}: Der Wert eines maximalen Flusses ist gleich der Kapazität eines minimalen (s,t)-Schnittes. Beweis durch Korrektheit des Labeling-Algorithmus.\\
\textbf{Satz (Augmenting-Path-Theorem}: Ein Fluss x ist maximal genau dann, wenn es im Restnetzwerk G(x) keinen erhöhenden Pfad gibt.\\
Beweis APT: Falls es einen erhöhenden Pfad gibt, ist x nicht maxmial. Rückrichtung: $\not\exists$ erhöhender Pfad, führe Labeling aus, dann bekommt man einen (s,t)-Schnitt $[S,\overline{S}]$ mit S gelabeled und $\overline{S}$ nicht labeled, mit $F(x)=u[S,\overline{S}]$\\
\textbf{Satz (Ganzzahligkeit)}
Wenn alle Kapazitäten ganzzahlig, dann existiert ein maximaler Fluss x mit $x_{i,j}\in \mathbb{N}$ für all $(i,j)\in E$\\
\subsubsection*{Laufzeitanalyse}
Sei $U:= max\{u_{ij}| (i,j)\in E\}$ obere Schranke für $F_{max}$. Wir wissen $F_{max}\leq u[S,\overline{S}]$ für ale (s,t)-Schnitte.\\
Beobachtung: Höchstens n-1 Kanten verlaufen über den Schnitt (von s nach t). $\Rightarrow F_{max} \leq n*U$. In jeder Iteration von AUGMENT erhöht der Algorithmus den Flusswert F um mindestens 1. Daraus folgt: weniger als $F_{max}$ Iterationen $\leq n*U$. \\
Eine Iteration (Labeling) kostet Zeit \oh{n+m} (Wie DFS/BFS etc.) $\Rightarrow$ Gesamtlaufzeit \oh{n^2U+nmU}. Annahme G ist zusammenhängend $\Rightarrow m\geq n-1 \rightarrow$ \oh{nmU}

PREFlow (excess):$e(i)=\sum x_{ji} - \sum x_{ik} \leq 0$\\
Wenn $e(i)>0$ dann ist i aktiv ($i\not\in \{s,t\}$\\
Idee für Algorithmus:\\
1. Schiebe (push) Fluss von s nach t\\
2. Während des Ablaufs ensteht dadurch ein Preflow mit ExcessKnoten\\
3. am Ende soll der Preflow ein echter Fluss sein, d.h. keine aktiven Knoten\\
Für die Richtung s nach t verwenden wir eine Distanzfunktion (Distanz=Länge von Pfaden von s nach t)\\
\textbf{Definition: (Distanz-Label):}\\
Für einen Preflow x definieren wir in G(x) eine Distanzfunktion $d: V\rightarrow \mathbb{N}_0$ mit:
\begin{itemize}
\item[1.]d(t) = 0
\item[2.]$d(i)\leq d(j)+1$ für alle $(i,j)\in G(x)$
\end{itemize}
Beobachtung: 1. d ist eine untere Schranke für exakte Distanzwerte. 2. Die Nullfunktion ist ein gültiges Distanz-Label.\\
Intuition: d = Höhe/Level von Knoten\\
\textbf{Definition: admissible/zulässig}\\
Eine Kante (i,j) in G(x) heißt admissible gdw $d(i)=d(j)+1$\\
Preflow-Push Algorithmus:\\
1. Saturiere alle von s ausgehenden Kanten\\
2. Setze $d(i) = 0 $für $i\neq 0$ oder n für $i=s$\\
3. Betrachte einen aktiven Knoten i und schiebe Fluss über eine admissible Edge.\\
4. Falls ein aktiver Knoten i keine ausgehende admissible Edge hat -> Relabel $d(i)\leftarrow min\{d(j)|\text{j Nachbar in G(x)}\}+1$\\
Der generische Preflow-Push-Algorithmus:\\
\begin{lstlisting}[numbers=left,escapechar=!]
x=0;e=0;d=0;
forall j in V mit (s,j) in E do
	x_sj = u_sj //Volle Kapazität
	e(j) += x_sj
od
d(s) = n
while !$\exists$! aktive Knoten do
	wähle einen aktiven Knoten i //am besten highest label
	PUSH/RELABEL(i)
od
\end{lstlisting}
PUSH/RELABEL(knoten i):
\begin{lstlisting}[numbers=left,escapechar=!]
if !$\exists $! admissible Edge (i,j) in G(x) then
	wähle eine solche Kante (i,j)
	delta = min{e(i),r_ij}
	x_ij += delta
	e(i) -= delta
	e(j) += delta
else //Relabel
	d(i) = min{d(j)|(i,j) in G(x)}+1
fi
\end{lstlisting}
Laufzeit: \oh{\#Push+\#Relabel}\\
Lemma: Falls $e(i)>0$ dann existiert ein Pfad von i nach s im Restnetzwerk G(x).\\
Lemma2: Für alle aktiven Knoten $i\in V: d(i)<2n$ folgt aus Lemma 1.\\
Lemma3: Maximal $2n^2$ Realbels (Folgt aus Lemma2). Für jeden einzelnen Knoten maximal 2n.\\
Lemma4: $\leq n*m$ saturierende Pushes\\
Lemma5: \oh{n^2*m} nicht saturierende Pushes\\
Der Generische Algorithmus hat\oh{n^2*m}\\

Feasible Flow:\\
$G=(V,E)$ Gerichteter Graph\\
Kapazität $u_{ij}$ für alle Kanten\\
Supply-Werte b(i) für alle Knoten\\
$b(i)=\lbrace >0, $supply-Knoten,$=0, $Transport-Knoten, $<0,$ Demand-Knoten.\\
FeasibleFlow $x:E\rightarrow\mathbb{N}_0$ muss Bedingungen erfüllen:\\
1. $0\leq x_{ij} \leq u_{ij}$ für alle Kanten\\
2. Massenbalance: $\forall i\in V: b(i)=\sum x_{ij} - \sum x_{ki}$ mit (i,j) ausgehende und (k,i) eingehende Kanten.\\
Offensichtlich: $\sum_{i\in V} b(i) = 0$\\
Berechne Maxflow für dieses Netzwertk -> x*\\
2 Fälle:\\
1. x* saturiert alle aus s ausgehen Kanten, und damit auch alle in t gehenden\\
Dann ist x* eingeschränkt auf Originalnetzwerk ein feasible flow.\\
2. Sonst (nicht alle durch Maxflow x* saturiert) -> Es existiert kein Feasible flow.

\end{document}

