\documentclass[a4paper]{article}
\usepackage{amsmath}
\usepackage{amssymb}
\usepackage{fancyhdr}
\usepackage{a4wide}
\usepackage{geometry}
\usepackage[utf8]{inputenc}
\usepackage{graphicx}
\geometry{a4paper,left=2cm,right=2cm, top=3cm, bottom=3cm}
\usepackage{listings}
\lstset{language=xml}
\newcommand{\oh}[1]{$\mathcal{O}(#1)$}
\newcommand{\titel}[1]{\fancyhead[C]{#1}}
\newcommand{\name}{\fancyhead[L]{Alexander Landmesser}}
\newcommand{\matrikel}{\fancyhead[R]{}}
\newcommand{\pl}{\hspace*{1cm}}
\lstset{literate=
  {á}{{\'a}}1 {é}{{\'e}}1 {í}{{\'i}}1 {ó}{{\'o}}1 {ú}{{\'u}}1
  {Á}{{\'A}}1 {É}{{\'E}}1 {Í}{{\'I}}1 {Ó}{{\'O}}1 {Ú}{{\'U}}1
  {à}{{\`a}}1 {è}{{\`e}}1 {ì}{{\`i}}1 {ò}{{\`o}}1 {ù}{{\`u}}1
  {À}{{\`A}}1 {È}{{\'E}}1 {Ì}{{\`I}}1 {Ò}{{\`O}}1 {Ù}{{\`U}}1
  {ä}{{\"a}}1 {ë}{{\"e}}1 {ï}{{\"i}}1 {ö}{{\"o}}1 {ü}{{\"u}}1
  {Ä}{{\"A}}1 {Ë}{{\"E}}1 {Ï}{{\"I}}1 {Ö}{{\"O}}1 {Ü}{{\"U}}1
  {â}{{\^a}}1 {ê}{{\^e}}1 {î}{{\^i}}1 {ô}{{\^o}}1 {û}{{\^u}}1
  {Â}{{\^A}}1 {Ê}{{\^E}}1 {Î}{{\^I}}1 {Ô}{{\^O}}1 {Û}{{\^U}}1
  {œ}{{\oe}}1 {Œ}{{\OE}}1 {æ}{{\ae}}1 {Æ}{{\AE}}1 {ß}{{\ss}}1
  {ű}{{\H{u}}}1 {Ű}{{\H{U}}}1 {ő}{{\H{o}}}1 {Ő}{{\H{O}}}1
  {ç}{{\c c}}1 {Ç}{{\c C}}1 {ø}{{\o}}1 {å}{{\r a}}1 {Å}{{\r A}}1
  {€}{{\EUR}}1 {£}{{\pounds}}1
}
\lstset{language=c++}
\begin{document}
\title{Netzwerkalgorithmen}
\maketitle
\section{Topologisches Sortieren}
\subsection{BFS}
\begin{lstlisting}
bool TOPSORT (const graph& G, node_array<int>& topnum){
	int count = 0
	list<node> zero
	node_array<int> indeg(G)
	node v
	forall_nodes(v,G){
		indeg[v] = G.indeg(v)
		if (indeg[v] == 0) zero.append(v)
	}
	while (!zero.empty()){
		node v = zero.pop()
		topnum[v] = ++count
		edge e
		forall_out_edges(e,v){
			node w = G.target(e)
			if(--indeg[w] == 0) zero.append(w)
		}
	}
	return count == G.number_of_nodes()
}
\end{lstlisting}
\subsection{DFS}
\begin{lstlisting}
bool DFS_TOPSORT (const graph& G, node_array<int>& dfsnum, node_array<int>& compnum){
	int count1 = 0
	int count2 = 0
	node_array<bool> visited(G,false)
	node v
	forall_nodes(v,G){
		if (!visited[v]) dfs(G,v,count1,count2,dfsnum,compnum)
	}
}
void dfs(const graph& G, node v, int& count1, int& count2, node_array<int>& dfsnum, node_array<int>& compnum){
	dfsnum[v] = ++count1
	visited[v] = true
	edge e
	forall_out_edges(e,v){
		edge w = G.target(e)
		if(!visited[w]) dfs(G,w,count1,count2,dfsnum,compnum)
	}
	compnum[v] = ++count2
}
\end{lstlisting}
\subsection{Starke Zusammenhangskomponenten}
\section{Kürzeste Wege}
\subsection{\oh{n+m}}
\begin{lstlisting}
forall v in V do
	DIST[v] = unendlich
od
DIST[s] = 0
U = {s}
while U nicht leer
	wähle das u aus U mit minimaler topologischen Nummer
	forall v in V mit (u,v) in E do
		c = DIST[u]+cost(u,v)
		if c < DIST[v] then
			DIST[v] = c
			U = U + {v}
		fi
	od
od

\end{lstlisting}
\subsection{Dijkstra}
1x Konstruktor, n*(insert+delmin+empty), mx*decrease\\
Binärer Heap/Balancierter Baum: \oh{(n+m)*log(n)}\\
Fib-Heap: \oh{n*log(n)+m}
\begin{lstlisting}
void DIJKSTRA(const graph& G, node s, const edge_array<int>& cost, 
	_ const node_array<int>& dist, node_array<edge>& pred){
	node_pq<int> PQ(G);
	node v
	forall_nodes(v,G){
		dist[v] = unendl
		pred[v] = null
	}
	dist[s] = 0
	PQ.insert(s,0)
	while(!PQ.empty()){
		node u = PQ.delmin()
		edge e
		forall_out_edges(e,u){
			node v = G.target(e)
			int c = DIST[u]+cost(e)
			if(c < dist[v]){
				if(dist[v] == MAXINT){
					PQ.insert(v,c)
				}else{
					PQ.decrease(v,c)
				}
				dist[v] = c
				pred[v] = u
			}
		}
	}
}
\end{lstlisting}
\subsection{Bellman-Ford}
\begin{lstlisting}
bool BELLMAN(const graph& G, node s, const edge_array<int>& cost, node_array<int>& DIST){
	queue<node> Q;
	node_array<bool> inQ(G,false)
	node_array<int> count
	DIST[s] = 0
	Q.append(s)
	inQ[s] = true
	while (!Q.empty()){
		node u = Q.pop()
		inQ[u] = false
		if (++count[u] > G.number_of_nodes()){
			return false
		}
		edge e
		forall_out_edges(e,u){
			node v = G.target(e)
			int c = DIST[u] + cost[e]
			if (c<DIST[v]) {
				DIST[v] = c
				//Setze PRED verweis hier falls nötig
				if (!inQ[v]){
					Q.append(v)
					inQ[v] = true
				}
			}
		}
	}
	return true
}
				
\end{lstlisting}
\section{Maximaler Fluss}
\subsection{MF-Labeling}
Labeling: \oh{n+m}, Gesamt: \oh{n^2U+nmU}, da max n-1 Kanten über den Schnitt laufen und damit $F_{max} \leq n*U$ mit U der Kapazutät der mächtigsten Kante.\\
Bei zusammenhängendem Graphen:\oh{nmU}
\begin{lstlisting}
void MF_Labeling (const graph& G, node s, node t, cost edge_array<int>& cap, 
	_ edge_array<int>& flow){
	list<node> L
	node_array<bool> labeled(G,false)
	node_arrray<edge> pred(G,null)
	while(true){
		labeled[s] = true
		L.append(s)
		while(!L.empty()) {
			node v = L.pop()
			edge e
			forall_out_edges(e,v){
				if ( flow[e]==cap[e]) continue
				node w = G.target(e)
				if (labeled[w]) true
				labeled[w]=true
				pred[w]=e
				L.append(w)
			}
			forall_in_edges(e,v){
				if (flow[e]==0) continue
				node w = G.source(e)
				if (labeled[w]) continue
				labeled w = true
				pred[w] = e
				L.append(w)
			}
			if (labeled[t]) L.clear()
		}
		if (labeled[t]) AUGMENT(G,s,t,pred,cap,flow)
		else break
	}
}

void AUGMENT (const graph& G, node s, node t, cost node_array<edge>& pred, edge_array<int>& cap, edge_array<int>& flow){
	int delta = MAXINT
	node v = t
	while (v != s){
		int r
		edge e = pred[v]
		if(v==G.source(e)){
			r= flow[e]
			v = G.target[w]
		}else{
			r = cap[e]-flow[e]
			v = G.source(e)
		}
		if (r<delta) delta =r
	}
	v = t
	while (v!=s){
		edge e = pred[v]
		if(v==G.source(e)){
			flow[e] -= delta
			v = G.target(e)
		}else{
			flow[e] += delta
			v = G.source(e)
		}
	}
}
\end{lstlisting}
\subsection{Capacity-Scaling}
Anzahl der Phasen $\leq logU$\\
Jede Phase führt max 2m Erhöhungen aus + labeling: \oh{2mm}\\
\oh{m^2*log(U)}
\begin{lstlisting}
In Labeling:
	if(cap[e]-flow[e] < delta) continue 
	statt: if(flow[e] == cap[e]) continue
	
	if (flow[e] < delta) continue 
	statt: if (flow[e]==0) continue

CAPACITY_SCALING (G,s,t,cap,flow,U){
	flow = 0 //alle Flüsse = 0
	delta = 2^(log(U))
	while delta > 0 do
		MF-LABELING(G,s,t,cap,flow,delta)
		delta = delta / 2
	od
}
\end{lstlisting}
\end{document}

