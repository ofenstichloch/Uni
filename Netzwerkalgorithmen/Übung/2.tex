\documentclass[a4paper]{article}
\usepackage{amsmath}
\usepackage{amssymb}
\usepackage{fancyhdr}
\usepackage{a4wide}
\usepackage{geometry}
\usepackage[utf8]{inputenc}
\usepackage{graphicx}
\geometry{a4paper,left=2cm,right=2cm, top=3cm, bottom=3cm}

\newcommand{\titel}[1]{\fancyhead[C]{#1}}
\newcommand{\name}{\fancyhead[L]{Alexander Landmesser}}
\newcommand{\matrikel}{\fancyhead[R]{Matrikelnummer: 1096552}}
\newcommand{\pl}{\hspace*{1cm}}
\pagestyle{fancy}
\begin{document}
\matrikel
\titel{Netzwerkalgo - Übung 1}
\name

4.1: Sei x ein maximaler Fluss:\\
a) $0\leq x_{ij}\leq u_{ij}$ für alle $(i,j)\in E$\\
b) Für alle $i\in V\setminus \{s,t\}: \sum x_{ij} = \sum x_{k,i}$ und $F:=\sum_{(s,i)\in E}x_{si} -\sum_{(j,s)\in E}x_{js}$ mit F maximal.\\
Wie findet man den passenden MinCut (s,t)-Schnitt in Zeit O(m)\\
Lösung: 1 Iteration (die Letzte) des Labeling-Algorithmus. S=alle gelabelten, $\overline{S}$=andere\\
4.2: Umwandlung eines nicht-ganzzahligen max Flusses x in ganzzahligen\\
Beobachtung: 1. Flussänderung entlang von Kreisen in G(x) möglich, ohne den Flusswert $F_{max}$ zu verändern.\\
2. Sei (i,j) eine Kante, sodass $x_{ij}$ nicht ganzzahlig $\Rightarrow$ mindestens eine Kante inzident zu i und mindestens eine inzident zu j in G(x) hat ebenfalls ein nicht ganzzahliges x (Massenbalance)\\
Alle dieser Kanten liegen auf einem Kreis. Starte auf einer Kante wo $x_{ij}$ nicht ganzzahlig, laufe über nicht ganzzahlige Nachbarkanten, bis Kreis geschlossen ist. Kreis besitzt Restkapazität $\delta \rightarrow$ Flusserhöhung entlang dieses Kreises.
\end{document}