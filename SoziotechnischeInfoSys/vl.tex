\documentclass[a4paper]{article}
\usepackage{amsmath}
\usepackage{amssymb}
\usepackage{fancyhdr}
\usepackage{a4wide}
\usepackage{geometry}
\usepackage[utf8]{inputenc}
\usepackage{graphicx}
\geometry{a4paper,left=2cm,right=2cm, top=3cm, bottom=3cm}
\usepackage{listings}
\lstset{language=xml}
\newcommand{\oh}[1]{$\mathcal{O}(#1)$}
\newcommand{\titel}[1]{\fancyhead[C]{#1}}
\newcommand{\name}{\fancyhead[L]{Alexander Landmesser}}
\newcommand{\matrikel}{\fancyhead[R]{}}
\newcommand{\pl}{\hspace*{1cm}}
\lstset{literate=
  {á}{{\'a}}1 {é}{{\'e}}1 {í}{{\'i}}1 {ó}{{\'o}}1 {ú}{{\'u}}1
  {Á}{{\'A}}1 {É}{{\'E}}1 {Í}{{\'I}}1 {Ó}{{\'O}}1 {Ú}{{\'U}}1
  {à}{{\`a}}1 {è}{{\`e}}1 {ì}{{\`i}}1 {ò}{{\`o}}1 {ù}{{\`u}}1
  {À}{{\`A}}1 {È}{{\'E}}1 {Ì}{{\`I}}1 {Ò}{{\`O}}1 {Ù}{{\`U}}1
  {ä}{{\"a}}1 {ë}{{\"e}}1 {ï}{{\"i}}1 {ö}{{\"o}}1 {ü}{{\"u}}1
  {Ä}{{\"A}}1 {Ë}{{\"E}}1 {Ï}{{\"I}}1 {Ö}{{\"O}}1 {Ü}{{\"U}}1
  {â}{{\^a}}1 {ê}{{\^e}}1 {î}{{\^i}}1 {ô}{{\^o}}1 {û}{{\^u}}1
  {Â}{{\^A}}1 {Ê}{{\^E}}1 {Î}{{\^I}}1 {Ô}{{\^O}}1 {Û}{{\^U}}1
  {œ}{{\oe}}1 {Œ}{{\OE}}1 {æ}{{\ae}}1 {Æ}{{\AE}}1 {ß}{{\ss}}1
  {ű}{{\H{u}}}1 {Ű}{{\H{U}}}1 {ő}{{\H{o}}}1 {Ő}{{\H{O}}}1
  {ç}{{\c c}}1 {Ç}{{\c C}}1 {ø}{{\o}}1 {å}{{\r a}}1 {Å}{{\r A}}1
  {€}{{\EUR}}1 {£}{{\pounds}}1
}
\begin{document}
\title{Soziotechnische Informationssysteme}
\maketitle
\section{Grundlagen}
Soziotechnische Systeme sind in allen Maßen Superlative. Von benutzten Systemen über Datenmengen bis hin zur Anwenderzahl.\\
Emergenz: Es geschehen unerwartete Reaktionen auf das Zusammenspiel vieler Faktoren\\
Soziale Systeme besitzen immer mehr und genauere Abbilder der Gesellschaft. Zum Beispiel Bewegungsstatistiken von Mobilfunkanbietern etc.\\
\section{Soziale Netzstrukturen}
\subsection{Six Degrees of Separation}
Soziale Graphen zeichnen sich durch Superhubs aus, Knoten mit sehr vielen Kanten.\\
Jeder Mensch ist im Schnitt über 6 Verbindungen mit jedem anderen verbunden.\\
Strong Ties sind im Sozialen Sinne sehr wichtig, bei sozialen Netzwerken nicht ausschlaggebend. Weak Ties bilden die Netzstrukturen. 
Sehr viele Kanten.
\subsection{Bacon Number}
Entfernung zu Kevin Bacon (Wer das Kevin Bacon Game mit ihm/jemanden der mit ihm gespielt hat)
\subsection{Triadic Closure}
Wenn A B kennt und B kennt C, dann ist es sehr wahrscheinlich, dass A auch C kennt
Clustering-Koeffizient:\\
$\forall$3 Konten, mit 2 kanten = Offen, mit 3 Kanten = geschlossen\\
Für alle Knoten: closed/(closed-open) = globaler Clusterkoeffizient\\
\subsection{Preferential Attachment}
to be filled
\section{Zusätzliches}

\subsection{XML RPC (Web services)}
Sowohl Interfacebeschreibung, als auch Netzwerkstandard basieren auf XML. 
\subsection{Relationale Datenbankensysteme}
Nicht einfach verteilbar, kein Performancegewinn bei Joins. In Realität maximal bis zu 10-20 nodes\\
\subsection{NoSQL Datenbanken}
Aggregation Beispiel: Spiel speichert seine Spieler redundant.\\
Updates in Schemalosen DB sind schwer.\\
Kostet relativ viel Zeit als basis, skaliert aber\\
Transaktion: Wenn mehrere Transaktionen bearbeitet werden gilt: Es gibt eine Sequentielle anordnung der Transaktionen, die zum neuen konsistenten Zustand führen.\\
ACID vs BASE\\
Aggregation, .,., Durability\\
Eventual consistency: Irgendwann wenn keine Updates mehr kommen entsteht konsistenz\\
\subsection{Distributed Hashtables}
Daten werden im Ring verteilt. Z.B. das erste Byte entscheidet den Server...\\
\subsection{CAP-Theorem}
KLAUSUR MEEP MEEP BIIIDUUUBIIIDU\\
CA: Vertikal, klassische Datenbanksysteme\\
CP: Konsistenz wichtig, Geldautomaten\\
AP: Verfügbarkeit, Viel Replikation, Inkonsistenzen tolerieren, DNS, NOSQL\\
\subsection{Datenbanktypen}
Key-Value: Größte Gruppe, Values opaque (DB weiß nicht was Value ist)\\
Key-Document: Value ist semi-strukturiert, z.B. JSON\\
Column-Family:Nicht normalisierte Matrizen, Google big table
\subsection{Polyglotte Persistenz}
Ein System verwendet verschiedene Datensysteme (relational, Nosql etc)\\
SQL+NOSQL = NEWSQL, SQL generiert NOSQL Datenbanken\\

\end{document}